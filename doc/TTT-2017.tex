\documentclass[preprint,9pt]{sigplanconf}
\usepackage[references]{agda}
\usepackage[english]{babel}
\usepackage{todonotes}
\usepackage{alltt, enumitem, color, url}
\usepackage{catchfilebetweentags}
\setlength\mathindent{0em}

% \setmainfont{XITS}
% \setmathfont{XITS Math}

\begin{document}

\newcommand{\AB}[1]{\AgdaBound{#1}}
\newcommand{\AD}[1]{\AgdaDatatype{#1}}
\newcommand{\AF}[1]{\AgdaFunction{#1}}
\newcommand{\AS}[1]{\AgdaSymbol{#1}}

\newcommand{\command}[1]{\colorbox{lightgray}{\texttt{#1}}}

\conferenceinfo{TTT '17}{January 15th, 2017, Paris, France}
\copyrightyear{2017}
\copyrightdata{978-1-nnnn-nnnn-n/yy/mm}
\copyrightdoi{nnnnnnn.nnnnnnn}

\title{agdARGS - Declarative Hierarchical Command Line Interfaces}
% \subtitle{Subtitle Text, if any}

\authorinfo{Guillaume Allais}
           {gallais@cs.ru.nl}
           {Radboud University Nijmegen}
\maketitle

\section{Introduction}

If functional programmers using statically typed languages (broadly)
in the Hindley-Milner tradition have taken to boasting 
``If it typechecks, ship it!'', the shared sentiment amongst the ones
using dependently typed languages seemed for a long while to be closer
to ``Once it typechecks, shelve it!''.

Over the years, there has been some outliers not only using Type Theory
as a theorem proving tool but also demonstrating the practical benefits
dependent types bring to the programmers' table. The nowadays classic
definition of \texttt{printf} in a type-safe
manner~\cite{augustsson1998cayenne} is a prime example. Other notable
contributions in that vein include for instance parser DSLs and
generators~\cite{danielsson2010total,stump2016book},
and interactive systems~\cite{brady2014resource,claret2015mechanical}.

When writing an application in Type Theory, it is reasonable to expect
the programmer to want to focus her attention on the core algorithms
i.e. the parts that can be fully certified, dealing with sanitised data
enriched with all sorts of fancy invariants. The wrapper code is not
necessarily terribly exciting in comparison and tends to be treated
more as an afterthought.
Tanter and Tabareau~(\citeyear{tanter2015gradual}) have developed a
nice library to facilitate the transition from weakly typed to strongly
typed data whilst maintaining type safety. This potentially removes one
layer of boilerplate.

Command line interfaces are another one of these layers of wrapper
code. We offer a solution: a dependently typed DSL for defining
declaratively hierarchical command line interfaces available at
\url{https://github.com/gallais/agdARGS}.

\section{Hierarchical Command Line Interfaces}
\label{hcli}

A hierarchical command line interface is defined by:

\begin{itemize}
\item A \textbf{description} explaining the command's purpose. It has no
influence on the implementation of the interface but is useful
documentation for the end-user.

\item A list of \textbf{subcommands}. They are themselves fully-fledged
commands the user gets access to by mentioning a keyword. This
makes it possible to give the interface a \emph{hierarchical}
structure. E.g. \command{git pull} accesses the subcommand
\texttt{git-pull} from the main \texttt{git} interface with the
keyword \texttt{pull}.

\item A list of \textbf{modifiers} for the current command. They can
be either \textbf{flags} one may set or \textbf{options} taking
a parameter.

\item Finally, strings which are neither subcommand keywords nor
modifiers are considered \textbf{arguments} of the command.
\end{itemize}

With our library, the programmer can simply get an interface by
specifying this structure. For instance\footnote{Unfortunately
we lack the space necessary to give an example of an interface
with subcommands}, a command similar to UNIX's \texttt{wc} can
be declared this way:

\begin{figure}[ht]
\ExecuteMetaData[WordCount.tex]{wordcount}
\caption{The \texttt{wc} command's interface}\label{wc-cli}
\end{figure}

\section{Implementation Details}

If the general structure of a command is set in stone, it cannot
be the case for its subcommands and modifiers: they will vary
from application to application in number and nature. This means
that we need to design a first class representation of (extensible)
records amenable to generic programming to deal with them.

\subsection{Extensible Records}

A record type is characterised by two things: a list of distinct
field names and a type associated to each one of these fields.
Given a decidable \emph{strict} order on the type of names, we can
make use of McBride's design principles~(\citeyear{mcbride2014keep})
to define a structure of lists sorted in strictly increasing order
and thus only containing distinct elements. These will be our lists
of names. The types associated to each one of these field names can
be collected in a right-nested tuple computed by recursion on the list.
A record value is then a right-nested tuple of values of the corresponding
types.

Because there are so many computations at the type level, the
unification machinery can get stuck on meta-variables introduced
by implicit arguments. It is crucial for the usability of the
library defining extensible records that some of these notions
(the fields' types and the record value itself) are wrapped in
an Agda record to guide the type inference.

The combinators \AF{\_∷=\_⟨\_} and \AF{⟨⟩} one can see in
Figure~\ref{wc-cli} are also crucial to the library's usability:
they make it possible to define the extensible record field by
field without having to pay attention to the underlying representation
where all the invariants are enforced.


\subsection{Commands as Rose Trees}

The structure described in Section \ref{hcli} is reminiscent of a
rose tree and it is indeed implemented as one. It should now be
folklore that rose trees benefit a lot from being defined as sized
types~\cite{abelminiagda}. It allows recursive traversals to weaponize
higher-order functions without having to spend a lot of efforts
appeasing the termination checker. An instance of such a higher order
function we use to great effect is the fold over an extensible record
of subcommands.

We made Size an implicit index so that it does not add any extra
overhead from the programmer's point of view in places where it
does not matter.

\section{Generic Programming over Interfaces}

The point of having a first order representation of Interfaces
is, just like for any deeply-embedded DSL~\cite{hudak1996building},
to be able to write generic program against this representation.

\subsection{Parsing}

The most important use case is to harness the Interface declaration
to make sense of the list of strings\footnote{These are usually
referred to as ``command line arguments'' e.g. in the specification
of \texttt{getArgs} in the Haskell 98 report's ``System Functions''
section. We refrain from using that expression to avoid confusion
with our Interface's notion of arguments} passed to the executable
called from the command line. The expected result of a successful
parse is a path down the hierarchical structure of the interface
selecting a subcommand together with a collection of recognized
modifiers and arguments specific to that subcommand. Dependent types
allow us to make this requirement explicit by indexing the path
over the command it corresponds to. We write \AF{ParsedCLI} \AB{c}
for the type of successful parses associated to the interface \AB{c}.
This parsing process can be decomposed in three successive phases:

\begin{enumerate}[wide, labelwidth=!, labelindent=0pt]
\item The \textbf{subcommand selection} phase goes down the hierarchical
interface picking subcommands based on the keywords provided by the
user. As soon as a modifier or an argument for the current command
in focus is found, the second phase starts.

\item The \textbf{modifier and arguments collection} phase now has
settled for a given subcommand and tries to parse each new string as
either one of its modifiers or, if that doesn't succeed, an argument.

\item At any point the string ``-{}-'' can make the parser switch to the
\textbf{argument collection} phase. It interprets each subsequent
string as an argument to the command in focus. It is useful when
arguments may look like modifiers e.g. \command{ls -l -{}- -l} lists
(\texttt{ls}) in a long listing format (\texttt{-l}) the information
about the file ``-l'' (\texttt{-{}- -l}).
\end{enumerate}

We provide the user with a combinator readily putting various
pieces together that should fit most use cases. Its takes an interface,
a continuation for a successful parse and returns an \texttt{IO}
computation:
\ExecuteMetaData[WordCount.tex]{withCLI}
Internally, \AF{withCLI} performs a call to Haskell's \texttt{getArgs},
attempts to parse the list of strings it got back, and either prints
the error to \texttt{stdout} if the parse failed or calls the
continuation otherwise.

It is currently very simple but fits our need. We can imagine
more elaborate variations on it. We could for instance ``patch''
on the fly the provided interface so that it supports all the
common flags for requesting help (e.g. \texttt{-h}, \texttt{-{}-help},
\texttt{-?}, etc.) and responds to them by displaying appropriate
usage information.

\subsection{Usage Information}

It is indeed possible to exploit the available knowledge about
the interface's hierarchical structure, the subcommands' names
and their associated modifiers to generically produce usage
information for the end-users' consumption. Our \AF{usage}
function traverses the interface tree in a depth-first manner:
it starts by recursively displaying all the subcommands (if any)
at an increased indentation level and then lists the modifiers
for the current command. Ran on the \texttt{wc}-like interface
described in Figure~\ref{wc-cli}, it yields the output in
Figure~\ref{wc-usage}.

\begin{figure}[ht]
\begin{alltt}
WordCount  Print each file's counts
  --help     Display help
  --version  Version Number
  -l         Newline count
  -w         Word count
\end{alltt}
\caption{Usage Information for the Interface in Fig. \ref{wc-cli}}
\label{wc-usage}
\end{figure}

\section{Current Limitations and Future Work}

Writing the continuation passed to \AF{withCLI} can be a bit
verbose when dealing with deeply nested interfaces. It ought
to be possible to define combinators that make it easier to
combine together small, self-contained subcommands each one
handling its own branch of the subcommands tree.

It is rather common for interfaces to allow the grouping of
flags which are one character long into compound flags (e.g.
\command{tar -xz} is understood as \command{tar -x -z})
or to use the remainder of a one character long option as
its parameter (e.g. \command{tar -xz -ffi} is understood as
\command{tar -xz -f fi}). One can even mix the two e.g.
\command{tar -xzffi}. The current parser does not handle
these shortcuts.

The usage information is generated in a rather crude manner
by putting raw strings together. A well-structured intermediate
format describing in a simple manner the dependencies between
blocks of text would be an ideal candidate for a refactoring.
Wadler's Prettier Printer~(\citeyear{wadler2003prettier}) is a
possible candidate.

Once the generation of usage information is well structured,
it would be interesting to be able to generate proper \texttt{man}
pages. An interesting problem to solve towards that goal is the
generation of compact yet informative examples of valid usages.

\bibliographystyle{abbrvnat}
\bibliography{TTT-2017}

\end{document}
